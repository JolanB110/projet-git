Ce projet a été l’occasion d’utiliser \texttt{GitHub} dans une situation réelle et de découvrir ses nombreuses fonctionnalités, comme les \textit{issues}, les \textit{submodules} ou encore les \textit{pull requests}.

Cette expérience nous a donné envie d’utiliser \texttt{GitHub} pour nos projets personnels. Le simple fait d’avoir une grille de contribution visible motive à s’impliquer davantage, à créer ou rejoindre des projets variés pour progresser et construire notre identité de futurs développeurs.

Si nous avions une petite chose à regretter, ce serait de ne pas avoir eu plus de temps pour travailler sur un projet concret. Bien que la rédaction de ce rapport ait été enrichissante, nous aurions aimé coder un vrai projet de développement pour nous plonger pleinement dans les pratiques des développeurs professionnels.

Cela dit, ce n’est que le début : nous aurons de nombreuses occasions tout au long de nos études pour mettre en pratique tout ce que nous avons appris ici.